\documentclass[portrait,a0b,final,a4resizeable]{a0poster}
\usepackage{times}
\usepackage{graphicx}
\usepackage[usenames]{color}
\usepackage{multicol}
\usepackage{pstricks,pst-grad}
\usepackage{amsmath,amssymb}
\usepackage{amscd}
\usepackage{color}
\usepackage{fontspec}
\usepackage{tikz}
\usepackage{mathtools}
\usepackage[makeroom]{cancel}
\usepackage{etoolbox}

\usetikzlibrary{calc, fit, decorations.pathreplacing}
\thispagestyle{empty}
\patchcmd{\thebibliography}{\section*{\refname}}{}{}{}

\newgray{grispale}{0.7}
\newgray{grismoy}{0.5}
\newrgbcolor{lblue}{0.8 0.92 0.95}

\setlength{\columnsep}{5cm}
\setlength{\columnseprule}{2mm}
\setlength{\parindent}{0.0cm}

\newcommand{\background}[3]{
  \newrgbcolor{cgradbegin}{#1}
  \newrgbcolor{cgradend}{#2}
  \psframe[fillstyle=gradient,gradend=cgradend,
  gradbegin=cgradbegin,gradmidpoint=#3](0.,0.)(1.\textwidth,-1.\textheight)
}

\newenvironment{poster}{
  \begin{center}
  \begin{minipage}[c]{0.80\textwidth}
}{
  \end{minipage}
  \end{center}
}

\newcommand{\pbox}[4]{
\psshadowbox[#3]{
\begin{minipage}[t][#2][t]{#1}
#4
\end{minipage}
}}

\newcommand{\mainTitle}[1]{\begin{center}\pbox{0.5\columnwidth}{}{linewidth=2mm,framearc=0.1,linecolor=violetish,fillstyle=gradient,gradangle=0,gradbegin=white,gradend=whiteblue,gradmidpoint=1.0,framesep=1em}{\begin{center}\textbf{\huge #1}\end{center}}\end{center}\vspace{1.5cm}}
\newcommand{\subTitle}[1]{\begin{center}\pbox{0.5\columnwidth}{}{linewidth=2mm,framearc=0.1,linecolor=violetish,fillstyle=gradient,gradangle=0,gradbegin=white,gradend=whiteblue,gradmidpoint=1.0,framesep=1em}{\begin{center}\textbf{\LARGE #1}\end{center}}\end{center}\vspace{1.5cm}}
\renewcommand\CancelColor{\color{red}}
\newcommand{\skipLine}[1][1cm]{\vspace*{#1}\\}

\begin{document}
\newrgbcolor{blue}{0. 0. 0.80}
\newrgbcolor{violetish}{0.65, 0.66, 0.91}
\newrgbcolor{white}{1. 1. 1.}
\newrgbcolor{whiteblue}{.80 .80 1.}

\begin{poster}

\begin{center}
  \mainTitle{Lecture de séquences ADN: \textit{K}-mers et \textit{K}-mers espacés}
  \vspace*{1cm}
  \begin{tabular}{ccccc}
    \begin{minipage}[b]{0.2\textwidth}\centering\includegraphics[height=50mm]{logo_normandie.png}\end{minipage} &
    \begin{minipage}[b]{0.2\textwidth}\centering\includegraphics[height=50mm]{logo_univ_rouen.png}\end{minipage} &
    \begin{minipage}[b]{0.2\textwidth}
      \centering
      \textsl{\LARGE Gabriel Toublanc\\Master IGIS ITA\bigskip\\2015 - 2017}
    \end{minipage} &
    \begin{minipage}[b]{0.2\textwidth}\centering\includegraphics[height=50mm]{logo_ufr_sciences.png}\end{minipage} &
    \begin{minipage}[b]{0.2\textwidth}\centering\includegraphics[height=45mm]{logo_litis.png}\end{minipage}
  \end{tabular}\end{center}
  \vspace*{1cm}
\vspace*{3cm}

\begin{multicols}{2}
  \large \subTitle{Introduction}
    Les séquenceurs ADN ont connu un développement important depuis le milieu des années 2000, ce qui a permis à la bioinformatique de développer de nouvelle technique d'analyse.\skipLine
    Le séquençage ADN est constitué de deux principales familles de lectures :\\
    \begin{itemize}
      \setlength{\itemindent}{30mm}
      \renewcommand{\labelitemi}{$\longrightarrow$}
      \item \textbf{Les lectures courtes} (Illumina, Roche, ...)
      \item \textbf{Les lectures longues} (Nanopore, PacBio, ...)
    \end{itemize}\vspace*{1cm}
    Les \textbf{lectures courtes} ont le net avantage d'être \textbf{très précises}, avec un taux d'erreur de moins de 1 \%, la plupart étant des erreurs de substitution. Leur principal inconvénient est que \textbf{de nombreuses lectures} courtes sont nécessaire afin de reconstruire le génome, ce qui est coûteux et peu pratique.\skipLine[0.3cm]

    Les \textbf{lectures longues}, en revanche, ont l'avantage de ne nécessiter que d'\textbf{une petite quantité} d'entre elles pour couvrir le génome d'un être vivant, mais le gros désavantage d'être \textbf{hautement imprécises} (15 à 30 \% d'erreurs), principalement des erreurs d'insertions et de délétion.\skipLine[0.3cm]

    Dans les faits, le génome peut être reconstruit grâce à un nombre déraisonnable de lectures courtes, qui produiront un génome trop fragmenté, ou en assemblant les lectures longues corrigées par les lectures courtes. Plusieurs \textbf{méthodes de corrections} ont été développées à ce sujet \cite{Morisse2017}.\skipLine[0.3cm]

    L'étude des \textbf{\textit{k}-mers} des lectures longues a pour but de pouvoir corriger les lectures longues par elles-mêmes, et ainsi de pouvoir se passer des lectures courtes.\skipLine
  \subTitle{\textit{K}-mers}
    Un \textbf{\textit{k}-mer} est un facteur de taille $\mathbf{k}$ d'une séquence ADN donnée. Pour une séquence de taille $\mathbf{L}$, on a donc $\mathbf{L - \mathit{k} + 1}$ \textit{k}-mers possibles.\skipLine
    \pbox{0.9\columnwidth}{}{linewidth=2mm,framearc=0.1,linecolor=lblue,fillstyle=gradient,gradangle=0,gradbegin=white,gradend=white,gradmidpoint=1.0,framesep=1em}{
      \textbf{Ex:} Avec la séquence $\mathbf{s= AACCGGTT}$, on obtient les $k$-mers de taille $6$ ($6$-mers) suivants:\\
      \begin{center}\begin{tikzpicture}
        \coordinate (s) at (0, 6);
        \foreach \xi [count=\i] in {\mathbf{k_1:}, A, A, C, C, G, G, T, T} {
          \node[minimum size=15mm] (x\i) at (s) {$\xi$};
          \coordinate (s) at ($(s) + (2.5, 0)$);
        }
        \coordinate (s) at (0, 3);
        \foreach \yi [count=\i] in {\mathbf{k_2:}, A, A, C, C, G, G, T, T} {
          \node[minimum size=15mm] (y\i) at (s) {$\yi$};
          \coordinate (s) at ($(s) + (2.5, 0)$);
        }
        \coordinate (s) at (0, 0);
        \foreach \zi [count=\i] in {\mathbf{k_3:}, A, A, C, C, G, G, T, T} {
          \node[minimum size=15mm] (z\i) at (s) {$\zi$};
          \coordinate (s) at ($(s) + (2.5, 0)$);
        }
        \node[draw,rectangle,red,fit=(x2) (x7)] {};
        \node[draw,rectangle,red,fit=(y3) (y8)] {};
        \node[draw,rectangle,red,fit=(z4) (z9)] {};
      \end{tikzpicture}\end{center}
      \textbf{6-mers:} \{AACCGG, ACCGGT, CCGGTT\}
    } \skipLine
    Afin d'énumérer les \textit{k}-mers de grands jeux de données, divers outils logiciels ont été développés. L'objectif de ces outils est d'extraire les \textit{k}-mers, \textbf{leur nombre d'occurences}, d'avoir un \textbf{temps d'exécution} viable et d'utiliser le moins d'\textbf{espace mémoire} possible.\skipLine

    Le logiciel \textbf{Jellyfish} \cite{Marcais2011} surpasse ses concurrents dans ce domaine, avec des performances bien supérieures, tout facteur confondu.\skipLine

    Les \textit{k}-mers sont principalement utilisés pour \textbf{l'alignement et l'assemblage} de lectures.\skipLine

    Dans le cas présent, l'utilisation des \textit{k}-mers a pour but de \textbf{trouver des répétitions} au sein des lectures longues afin de pouvoir \textbf{identifier les nucléotides erronées}, et de les corriger.
  \newpage
  \subTitle{\textit{K}-mers espacés à délétion}
    Les \textbf{\textit{k}-mers espacés} sont utilisés afin de simuler des corrections aux erreurs d'insertions et de délétions sur les lectures longues. Au lieu de prendre les facteurs de taille \textit{k} d'une séquence, on va sélectionner les \textit{k}-mers espacés selon un motif précis.\skipLine

    Avec les \textbf{\textit{k}-mers espacés à délétion}, chaque $\mathbf{0}$ du motif correspond à un nucléotide à supprimer.\\
    Par ex, un motif $\mathbf{m=111011}$ permettra d'extraire tous les $5$-mers espacés à partir des $6$-mers en supprimant le $4^{eme}$ nucléotide.\skipLine
    \pbox{0.9\columnwidth}{}{linewidth=2mm,framearc=0.1,linecolor=lblue,fillstyle=gradient,gradangle=0,gradbegin=white,gradend=white,gradmidpoint=1.0,framesep=1em}{
      \textbf{Ex:} Avec la séquence $\mathbf{s = AACCGGTT}$ et le motif $m = 111011$, on obtient les $5$-mers espacés suivants:\skipLine
      \begin{center}\begin{tikzpicture}
        \coordinate (s) at (0, 6);
        \foreach \xi [count=\i] in {\mathbf{k_1:}, A, A, C, \xcancel{C}, G, G, T, T} {
          \node[minimum size=15mm] (x\i) at (s) {$\xi$};
          \coordinate (s) at ($(s) + (2.5, 0)$);
        }
        \coordinate (s) at (0, 3);
        \foreach \yi [count=\i] in {\mathbf{k_2:}, A, A, C, C, \xcancel{G}, G, T, T} {
          \node[minimum size=15mm] (y\i) at (s) {$\yi$};
          \coordinate (s) at ($(s) + (2.5, 0)$);
        }
        \coordinate (s) at (0, 0);
        \foreach \zi [count=\i] in {\mathbf{k_3:}, A, A, C, C, G, \xcancel{G}, T, T} {
          \node[minimum size=15mm] (z\i) at (s) {$\zi$};
          \coordinate (s) at ($(s) + (2.5, 0)$);
        }
        \node[draw,rectangle,red,fit=(x2) (x4)] {};
        \node[draw,rectangle,red,fit=(x6) (x7)] {};
        \node[draw,rectangle,red,fit=(y3) (y5)] {};
        \node[draw,rectangle,red,fit=(y7) (y8)] {};
        \node[draw,rectangle,red,fit=(z4) (z6)] {};
        \node[draw,rectangle,red,fit=(z8) (z9)] {};
      \end{tikzpicture}\end{center}
      \textbf{5-mers:} \{AACGG, ACCGT, CCGTT\}
    } \skipLine[0.1cm]
    \subTitle{\textit{K}-mers espacés à insertion}
      Avec les \textbf{\textit{k}-mers espacés à insertion}, chaque $\mathbf{0}$ du motif correspond à un nucléotide à insérer.\\
      Tous les nucléotides possibles pour chaque $\mathbf{0}$ sont ajoutés, ce qui produit $\mathbf{4^t}$ fois plus \textit{k}-mers, avec $\mathbf{t = }$nombre de zéros dans le motif.\skipLine
      \pbox{0.9\columnwidth}{}{linewidth=2mm,framearc=0.1,linecolor=lblue,fillstyle=gradient,gradangle=0,gradbegin=white,gradend=white,gradmidpoint=1.0,framesep=1em}{
        \textbf{Ex:} Avec la séquence $\mathbf{s = AACCGGTT}$ et le motif $\mathbf{m = 111011}$, on obtient les $6$-mers espacés suivants:\skipLine
        \begin{center}\begin{tikzpicture}
          \coordinate (s) at (0, 9);
          \foreach \xi [count=\i] in {$\mathbf{k_1:}$, $A$, $A$, $C$, \small$\underbrace{A, C, G, T}_{}$, $C$, $G$, $G$, $T$, $T$} {
            \node[minimum size=15mm] (x\i) at (s) {\xi};
            \coordinate (s) at ($(s) + (2.5, 0)$);
          }
          \coordinate (s) at (0, 6);
          \foreach \yi [count=\i] in {$\mathbf{k_2:}$, $A$, $A$, $C$, $C$, \small$\underbrace{A, C, G, T}_{}$, $G$, $G$, $T$, $T$} {
            \node[minimum size=15mm] (y\i) at (s) {\yi};
            \coordinate (s) at ($(s) + (2.5, 0)$);
          }
          \coordinate (s) at (0, 3);
          \foreach \zi [count=\i] in {$\mathbf{k_3:}$, $A$, $A$, $C$, $C$, $G$, \small$\underbrace{A, C, G, T}_{}$, $G$, $T$, $T$} {
            \node[minimum size=15mm] (z\i) at (s) {\zi};
            \coordinate (s) at ($(s) + (2.5, 0)$);
          }
          \coordinate (s) at (0, 0);
          \foreach \mi [count=\i] in {$\mathbf{k_4:}$, $A$, $A$, $C$, $C$, $G$, $G$, \small$\underbrace{A, C, G, T}_{}$, $T$, $T$} {
            \node[minimum size=15mm] (m\i) at (s) {\mi};
            \coordinate (s) at ($(s) + (2.5, 0)$);
          }
          \node[draw,rectangle,red,fit=(x2) (x7)] {};
          \node[draw,rectangle,red,fit=(y3) (y8)] {};
          \node[draw,rectangle,red,fit=(z4) (z9)] {};
          \node[draw,rectangle,red,fit=(m5) (m10)] {};
        \end{tikzpicture}\end{center}
        \textbf{6-mers:} $\left\{\text{\begin{tabular}{llll}
          AAC\textbf{A}CG, &ACC\textbf{A}GG, &CCG\textbf{A}GT, &CGG\textbf{A}TT,\\
          AAC\textbf{C}CG, &ACC\textbf{C}GG, &CCG\textbf{C}GT, &CGG\textbf{C}TT,\\
          AAC\textbf{G}CG, &ACC\textbf{G}GG, &CCG\textbf{G}GT, &CGG\textbf{G}TT,\\
          AAC\textbf{T}CG, &ACC\textbf{T}GG, &CCG\textbf{T}GT, &CGG\textbf{T}TT\\
        \end{tabular}}\right\}$
      }\skipLine[0.5cm]
  \subTitle{Références}
    \bibliographystyle{unsrt}
    \bibliography{biblio}
  \newpage
\end{multicols}
\end{poster}
\end{document}
