\documentclass[11pt]{beamer}

\usetheme[progressbar=frametitle]{metropolis}
\usepackage[french]{babel}
\usepackage{datetime, xcolor, amsthm, tikz}
\usepackage[makeroom]{cancel}
\usepackage[french]{algorithm2e}

\usetikzlibrary{calc, fit, decorations.pathreplacing}

\title{Memoire de stage de Maîtrise IGIS ITA}
\subtitle{K-mers et k-mers espacés}
\date{\formatdate{26}{01}{2017}}
\author{Gabriel Toublanc}
\institute{Université de Rouen, U.F.R des Sciences et Techniques de Saint-Etienne-du-Rouvray, Equipe LITIS TIBS}

\newcommand{\pauseline}{\\\pause\bigskip}
\newcommand{\setcolor}[2]{\textbf{\textcolor{#1}{#2}}}
\newcommand{\red}[1]{\setcolor{red}{#1}}
\newcommand{\green}[1]{\setcolor{green}{#1}}
\newcommand{\blue}[1]{\setcolor{blue}{#1}}

\newtheorem{thmgt}{Théorème}
\newtheorem{defgt}{Définition}
\newtheorem{lemgt}{Lemme}
\newtheorem{corgt}{Corollaire}
\newtheorem{progt}{Proposition}

\renewcommand\qedsymbol{$\blacksquare$}

\begin{document}
\maketitle
\metroset{block=fill}

\section{Introduction}
\begin{frame}[fragile]{Introduction}
\end{frame}

\section{Comptage des k-mers}
\begin{frame}[fragile]{K-mers}
  Les k-mers sont des facteurs de séquences ADN.\\
  \pause
  \textbf{Ex:} Avec une séquence $x = AACCGGTT$, on a les $k$-mers de taille $6$ ($6$-mers) suivants:\\
  \pause
  \begin{center}\begin{tikzpicture}
    \coordinate (s) at (0, 2);
    \foreach \xi [count=\i] in {\mathbf{k_1:}, A, A, C, C, G, G, T, T} {
      \node[minimum size=5mm] (x\i) at (s) {$\xi$};
      \coordinate (s) at ($(s) + (0.8, 0)$);
    }
    \node[draw,rectangle,red,fit=(x2) (x7)] {};
    \pause
    \coordinate (s) at (0, 1);
    \foreach \yi [count=\i] in {\mathbf{k_2:}, A, A, C, C, G, G, T, T} {
      \node[minimum size=5mm] (y\i) at (s) {$\yi$};
      \coordinate (s) at ($(s) + (0.8, 0)$);
    }
    \node[draw,rectangle,red,fit=(y3) (y8)] {};
    \pause
    \coordinate (s) at (0, 0);
    \foreach \zi [count=\i] in {\mathbf{k_3:}, A, A, C, C, G, G, T, T} {
      \node[minimum size=5mm] (z\i) at (s) {$\zi$};
      \coordinate (s) at ($(s) + (0.8, 0)$);
    }
    \node[draw,rectangle,red,fit=(z4) (z9)] {};
  \end{tikzpicture}\end{center}
\end{frame}
\begin{frame}[fragile]{K-mers espacés à délétion}
  Utilisé afin de simuler les erreurs d'insertions sur les lectures longues.\\
  \pause
  \textbf{Ex:} Avec une séquence $x = AACCGGTT$ et le motif $m = 111011$, on a les $5$-mers espacés suivants:\\
  \pause
  \begin{center}\begin{tikzpicture}
    \coordinate (s) at (0, 2);
    \foreach \xi [count=\i] in {\mathbf{k_1:}, A, A, C, \xcancel{C}, G, G, T, T} {
      \node[minimum size=5mm] (x\i) at (s) {$\xi$};
      \coordinate (s) at ($(s) + (0.8, 0)$);
    }
    \node[draw,rectangle,red,fit=(x2) (x4)] {};
    \node[draw,rectangle,red,fit=(x6) (x7)] {};
    \pause
    \coordinate (s) at (0, 1);
    \foreach \yi [count=\i] in {\mathbf{k_2:}, A, A, C, C, \xcancel{G}, G, T, T} {
      \node[minimum size=5mm] (y\i) at (s) {$\yi$};
      \coordinate (s) at ($(s) + (0.8, 0)$);
    }
    \node[draw,rectangle,red,fit=(y3) (y5)] {};
    \node[draw,rectangle,red,fit=(y7) (y8)] {};
    \pause
    \coordinate (s) at (0, 0);
    \foreach \zi [count=\i] in {\mathbf{k_3:}, A, A, C, C, G, \xcancel{G}, T, T} {
      \node[minimum size=5mm] (z\i) at (s) {$\zi$};
      \coordinate (s) at ($(s) + (0.8, 0)$);
    }
    \node[draw,rectangle,red,fit=(z4) (z6)] {};
    \node[draw,rectangle,red,fit=(z8) (z9)] {};
  \end{tikzpicture}\end{center}
\end{frame}
\begin{frame}[fragile]{K-mers espacés à insertion}
  Utilisé afin de simuler les erreurs de délétion sur les lectures longues.\\
  \pause
  $\mathbf{(L - k + 1)*4^t}$ k-mers espacés à insertion possibles.\\
  \pause
  \textbf{Ex:} Avec une séquence $\mathbf{x = AACCGGTT}$ et le motif $\mathbf{m = 111011}$, on a les $6$-mers espacés suivants:\\
  \pause
  \begin{center}\begin{tikzpicture}
    \coordinate (s) at (0, 3);
    \foreach \xi [count=\i] in {$\mathbf{k_1:}$, $A$, $A$, $C$, \tiny$\underbrace{A, C, G, T}_{}$, $C$, $G$, $G$, $T$, $T$} {
      \node[minimum size=5mm] (x\i) at (s) {\xi};
      \coordinate (s) at ($(s) + (0.8, 0)$);
    }
    \node[draw,rectangle,red,fit=(x2) (x7)] {};
    \pause
    \coordinate (s) at (0, 2);
    \foreach \yi [count=\i] in {$\mathbf{k_2:}$, $A$, $A$, $C$, $C$, \tiny$\underbrace{A, C, G, T}_{}$, $G$, $G$, $T$, $T$} {
      \node[minimum size=5mm] (y\i) at (s) {\yi};
      \coordinate (s) at ($(s) + (0.8, 0)$);
    }
    \node[draw,rectangle,red,fit=(y3) (y8)] {};
    \pause
    \coordinate (s) at (0, 1);
    \foreach \zi [count=\i] in {$\mathbf{k_3:}$, $A$, $A$, $C$, $C$, $G$, \tiny$\underbrace{A, C, G, T}_{}$, $G$, $T$, $T$} {
      \node[minimum size=5mm] (z\i) at (s) {\zi};
      \coordinate (s) at ($(s) + (0.8, 0)$);
    }
    \node[draw,rectangle,red,fit=(z4) (z9)] {};
    \pause
    \coordinate (s) at (0, 0);
    \foreach \mi [count=\i] in {$\mathbf{k_4:}$, $A$, $A$, $C$, $C$, $G$, $G$, \tiny$\underbrace{A, C, G, T}_{}$, $T$, $T$} {
      \node[minimum size=5mm] (m\i) at (s) {\mi};
      \coordinate (s) at ($(s) + (0.8, 0)$);
    }
    \node[draw,rectangle,red,fit=(m5) (m10)] {};
  \end{tikzpicture}\end{center}
\end{frame}

\section{KmersDel}
\begin{frame}[fragile]{KmersDel}
  \scriptsize
  \begin{algorithm}[H]{
    \Entree{table\_hachage $table$, chaînes $lectures$, chaîne $motif$, entier $k$}
    \pause
    \PourCh{$lecture$ de $lectures$}{
      \pause
      \Pour{$i = 0;\ i + k \le |lecture|;\ i += 1$}{
        $kmerEntier = 0$\;
        $kmer = ""$\;
        \pause
        \Pour{$j = 0;\ j < k;\ j += 1$}{
          \lSi{$motif[j] \ne 0$}{$kmer = kmer + lecture[i + j]$}
        }
        \pause
        \PourCh{$nucleotide$ de $kmer$}{
          $kmerEntier *= 4$\;
          \Suivant{valeur de $nucleotide$}{
            \lCas{A}{}
            \lCas{C}{$kmerEntier += 1$}
            \lCas{G}{$kmerEntier += 2$}
            \lCas{T}{$kmerEntier += 3$}
          }
        }
        \pause
        $table[kmerEntier] += 1$\;
      }
    }
  }\end{algorithm}
  \normalsize
\end{frame}

\section{KmersExpand}
\begin{frame}[fragile]{KmersExpand}
  \textbf{\normalsize kmerExpand}
  \pause
  \scriptsize
  \begin{algorithm}[H]{
    \Entree{chaînes $lectures$, chaîne $motif$, entier $k$}
    \pause
    \PourCh{$lecture$ de $lectures$}{
      \pause
      \Pour{$i = 0;\ i + k \le |lecture|;\ i += 1$}{
        $kmer = lecture[i:k]\;$
        $kmerExpandRec(kmer, motif, 0)\;$
      }
    }
  }\end{algorithm}
  \pause
  \textbf{\normalsize kmerExpandRec}
  \begin{algorithm}[H]{
    \Entree{chaîne $kmer$, chaîne $nvKmer$, chaîne $motif$, entier $posMotif$, entier $posKmer$}
    \pause
    \lSi{$posSeed == |motif|$}{$affiche(nvKmer)$}
    \pause
    \lSinonSi{$motif[posMotif] \ne 0$}{$kmersExpandRec(kmer, nvKmer + kmer[posKmer], posMotif+1, posKmer+1)$}
    \pause
    \Sinon{
      $kmersExpandRec(kmer, nvKmer + A, posMotif+1, posKmer)$\;
      $kmersExpandRec(kmer, nvKmer + C, posMotif+1, posKmer)$\;
      $kmersExpandRec(kmer, nvKmer + G, posMotif+1, posKmer)$\;
      $kmersExpandRec(kmer, nvKmer + T, posMotif+1, posKmer)$\;
    }
  }\end{algorithm}
  \normalsize
\end{frame}

\section{Mots Minimaux absents (MMA)}
\begin{frame}[fragile]{Mots Minimaux absents (MMA)}
\end{frame}

\section{Plus long sous-mot commun (PLS)}
\begin{frame}[fragile]{Plus long sous-mot commun (PLS)}
\end{frame}

\section{Résultats obtenus}
\begin{frame}[fragile]{Résultats: KmersDel et kmersExpand}
\end{frame}
\begin{frame}[fragile]{Résultats: Mots Minimaux absents (MMA)}
\end{frame}
\begin{frame}[fragile]{Résultats: Plus long sous-mot commun (PLS)}
\end{frame}

\begin{frame}[standout]
Conclusion
\end{frame}
\end{document}
