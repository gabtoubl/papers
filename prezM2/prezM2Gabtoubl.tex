\documentclass[11pt]{beamer}

\usetheme[progressbar=frametitle]{metropolis}
\usepackage[french]{babel}
\usepackage{datetime, xcolor, amsthm, tikz}

\usetikzlibrary{calc, decorations.pathreplacing}

\title{Memoire de stage de Maîtrise IGIS ITA}
\subtitle{K-mers et k-mers espacés}
\date{\formatdate{26}{01}{2017}}
\author{Gabriel Toublanc}
\institute{Université de Rouen, U.F.R des Sciences et Techniques de Saint-Etienne-du-Rouvray, Equipe LITIS TIBS}

\newcommand{\pauseline}{\\\pause\bigskip}
\newcommand{\setcolor}[2]{\textbf{\textcolor{#1}{#2}}}
\newcommand{\red}[1]{\setcolor{red}{#1}}
\newcommand{\green}[1]{\setcolor{green}{#1}}
\newcommand{\blue}[1]{\setcolor{blue}{#1}}

\newtheorem{thmgt}{Théorème}
\newtheorem{defgt}{Définition}
\newtheorem{lemgt}{Lemme}
\newtheorem{corgt}{Corollaire}
\newtheorem{progt}{Proposition}

\renewcommand\qedsymbol{$\blacksquare$}

\begin{document}
  \maketitle
  \metroset{block=fill}

  \section{Introduction}
    \begin{frame}[fragile]{Recherche exacte}
    \end{frame}

  \section{Encodage des mots}
    \begin{frame}[fragile]{Encodage des mots}
    \end{frame}

  \section{Quelques définitions}
    \begin{frame}[fragile]{Antécédent}
    \end{frame}

  \section{Théorèmes découverts}
    \begin{frame}[fragile]{Idée principale}
    \end{frame}

\begin{frame}[standout]
	Conclusion
\end{frame}

\end{document}
